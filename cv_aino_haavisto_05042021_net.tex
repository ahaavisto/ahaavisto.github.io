\documentclass{tccv_fin}
%\usepackage[finnish]{babel}
\usepackage{polyglossia} %tavallaan korvaavat babelin ja vaativat xelatex
\setdefaultlanguage{finnish} %tavallaan korvaavat babelin ja vaativat xelatex
\usepackage[pages=some]{background}
\usepackage{fancyhdr}

\backgroundsetup{
scale=1,
color=black,
opacity=0.4,
angle=0,
contents={%
  %\includegraphics[width=\paperwidth,height=\paperheight]{1401354894353.png}
  %alkuperäinen \includegraphics[width=\paperwidth,height=\paperheight]{example-image}
  }%
}
%\pagestyle{fancy}
%\fancyhf{} 
%\cfoot{footerteksti (joka ei näy)}

\begin{document}



\BgThispage

\part{Aino Haavisto}

\section{Työkokemus}

\begin{eventlist}

\item{1/2021--4/2021}
    {Aasian ja Afrikan kielten hanke}
    {\textbf{Oppikirjan toimittaja}}
    
Toimitin ja taitoin \textit{Omigoto! 7} ja \textit{Omigoto! 8} -oppikirjat lukion japanin opetukseen.

%\item{2019--}
%    {Ippo ippo -oppikirjahanke}
%    {\textbf{Oppikirjantekijä}}
    
%Työryhmämme tekee Opetushallituksen rahoituksella en\-sim\-mäistä suomalaista japanin kielen kirjoitusmerkkien oppikirjaa. Työ on sivutoimista ja valmistuu 2022.

\item{11/2020--12/2020}
     {Luonto-Liitto ry}
     {\textbf{Toiminnanjohtajan sijainen}}

Toiminnanjohtajan sijaisena koordinoin lasten ja nuorten ympäristöjärjestön kokonaisuutta kahden kuukauden 

ajan. Erityisesti hoidin sisäistä viestintää eri toimijoiden välillä ja järjestin kokouksia.

\item{11/2019--8/2020}
	 {Helsingin yliopisto}
	 {\textbf{Projektisuunnittelija}}
	 
Kehitin Verkosto-oppimisen pilotti Kumpulan kampuk\-sel\-la -projektissa uusia muotoja tukea jatkuvaa oppimista sekä luoda uusia yhteyksiä yliopiston ja muiden toi\-mi\-joi\-den, erityisesti yritysten, välille.

%\item{8/2019--5/2020}
%	 {Helsingin englantilainen koulu}
%	 {\textbf{Japani-kerhon ohjaaja}}
	 
%Vedin viikottaista alakoululaisille suunnattua kerhoa, jos\-sa tutustuttiin japanin kieleen ja kulttuuriin. Koulujen siir\-ryt\-tyä etäopetukseen tein loppukevään opetusvideoita ja tehtävä\-kokoelmia kerholaisille.

%\item{9/2018--5/2019}
%	 {Helsingin yliopisto}
%	 {\textbf{Tutkimusavustaja}}
	 
%Toimin tutkimusavustajana Ancient Near Eastern 


%Empires -huippuyksikössä mm. tiedeviestinnän ja tapah\-tu\-man\-järjestämisen parissa.


%\item{2015--2019}
%	 {Suomen kielitieteellinen yhdistys}
%	 {\textbf{Taittaja}}
%	 
%Taitoin Suomen kielitieteellisen yhdistyksen vuosittain ilmestyvää tieteellistä aikakauskirjaa.

\item{2014--2019}
	 {Tiedekasvatuskeskus, Linkki}
	 {\textbf{Tiedekasvattaja}}
	 
Ohjasin paljon erilaista tietojenkäsittelytieteeseen liit\-ty\-vää toimintaa lapsille, nuorille ja opettajille: ohjel\-moin\-ti\-ker\-hoja, työpajoja, oppimateriaaleja sekä li\-sä\-kou\-lu\-tin opet\-tajia. %Olen aktiivisesti kehittänyt Linkin käytäntöjä ja pyrkinyt tuomaan toimintaan entistä aktiivisemmin mukaan laajempaa tiedekasvatusaspektia.

%\item{2014, 2015}
%	 {University of Helsinki}
%	 {Lab assistant}
	 
%Worked as a course assistant in the courses Introduction to Programming, Advanced Programming and Computing tools for computer science studies.

%\item{2010--12/2020}
%     {Luonto-Liitto ry}
%     {\textbf{Useita toimia}}

%Lasten ja nuorten ympäristöjärjestö Luonto-Liitossa olen mm. järjestänyt lasten ja nuorten leirejä, kouluttanut uusia ympäristökasvattajia ja kehittänyt koulutusrunkoja sekä toiminut vuoden 2020 lopussa kaksi kuukautta toi\-min\-nanjohtajan sijaisena.

\item{2010--}
     {Luonto-Liitto ry}
     {\textbf{Ympäristökasvattaja ja kouluttaja}}

Olen toiminut Luonto-Liitossa mo\-nen\-lai\-sissa tehtävissä: viime vuosina eri\-tyi\-ses\-ti kouluttajana, mutta myös muun muassa luontoleirien johtajana, kerhon vetäjänä ja kok\-ki\-na. Olen kehittänyt olemassaolevia kou\-lu\-tus\-run\-koja ja vuonna 2020 siirtänyt etämuotoon työ\-pa\-joja ja kokouksia.


\end{eventlist}



\section{Koulutus}

\begin{yearlist}

\item[Japanin opintosuunta]{2016--2019}
     {Kielten maisteriohjelman FM, Helsingin yliopisto}
     {Sivuaine: assyriologia}
     
\item[Tietojenkäsittelytiede]{2014--2020}
     {Tietojenkäsittelytieteen LuK, Helsingin yliopisto}
     {Sivuaineet: matematiikka, assyriologia}
     
\item[Japanin opintosuunta]{2013--2016}
     {Aasian tutkimuksen HuK, 
     
     Helsingin yliopisto}
     {Sivuaineet: yleinen kielitiede, tietojenkäsittelytiede}
\end{yearlist}     


\personal %tästä tulis siis sitaatti!
    [ei toimi]
    {ei toimi}


\section{Luottamustoimet}
\begin{eventlist}

\item{2018--2020}
     {Luonto-Liitto ry}
     {\textbf{Varapuheenjohtaja}}
     
     Varapuheenjohtaja sekä liittohallituksen jäsen. Olen ollut kaudellani mukana kehittämässä laajasti järjestöä stra\-te\-gia\-työn kautta.

\item{2017--}
     {Suomen Itämainen Seura ry}
     {\textbf{Sihteeri}}
     
     Aasian, Afrikan ja Lähi-idän alueiden tutkimusta edis\-tä\-vän tieteellisen seuran sihteeri. Olen huo\-leh\-ti\-nut jä\-sen\-vies\-tin\-nästä, uudistanut seuran in\-ter\-net\-si\-vut ja ollut päävastuussa seuran useimmista ta\-pah\-tu\-mis\-ta.

\item{2016--}
     {Japanilaisen kulttuurin ystävät ry}
    {\textbf{Päätoimittaja}}
     
   	 Ystävyysyhdistyksen Tomo-lehden päätoimittajana ko\-koan lehteen tulevat tekstit, editoin ne julkaisukuntoon, valitsen kuvat taittajan kanssa ja kirjoitan itse lehteen.
     
\item{2013--2016}
     {Karavaani ry}
     {\textbf{Puheenjohtaja ja muita toimia}}
     
     Puheenjohtajuuden (2015) lisäksi olen toiminut Aasian, Afrikan ja Lähi-idän tutkimuksen opiskelijoiden aine\-jär\-jestössä rahanstonhoitajana ja viestintätehtävissä.%lehden päätoimittajana, internet-sivujen yllä\-pi\-täjänä ja rahastonhoitajana.
     
%\item{2014--2015}
%	 {Humanistinen tiedekunta}
%     {\textbf{Tiedekuntaneuvoston varajäsen}}     
     
\end{eventlist}


%\section{Apurahat}

%\begin{yearlist}

%\item{2017\hspace{3em}}
     %{Suomen assyriologisen tutkimuksen säätiön apuraha}
     %{Säätiön julkaisujärjestelmän teknisen puolen uudistamiseen yhtenä työryhmän jäsenenä.}
     
%\end{yearlist}

%\section{Muita toimia}

%\begin{yearlist}

%\item{2019--}
%    {Ippo ippo -oppikirjahanke}
%    {Työryhmämme tekee Opetushallituksen rahoituksella ensimmäistä suomalaista japanin kielen kirjoitusmerkkien oppikirjaa.}
     
%\item{2017--}
%	 {Liveroolipelien järjestäminen}
%	 {Olen järjestänyt useita täysipainoisia liveroolipelejä osallistuen sekä pelin suunnitteluun että tapahtumien käytännön toteutukseen.}

%\item{2014--}
 %    {Nuolenpäätyöpajoja}
  %   {Olen järjestänyt useita nuolen\-pää\-kir\-joi\-tus\-työpajoja eri kohdeyleisöille mm. yliopistolla, Tieteiden yössä ja kirjastossa.}

%\item{2011--}
 %    {Esitelmiä Japanista ja japanista}
  %   {Olen pitänyt yhteensä kuusi yleisöluentoa Japaniin liittyen erilaisissa tapahtumissa.}
     
%\end{yearlist}

\section{IT-taidot}

\begin{factlist}
\item{Viestintä}{Drupal, MS Office ja Office365, Wordpress}
\item{Muita teknologioita}{CSS, Git, HTML, Java, Javascript, Python}
\end{factlist}

\section{Kielitaito}

\begin{factlist}
\item{Äidinkieli}{suomi}
\item{Hyvä}{englanti}
\item{Keskitaso}{japani, ruotsi}
\item{Perustaidot}{saksa}
\item{Lukutaito}{akkadi, sumeri, klassinen japani}
\end{factlist}

\section{Harrastukset}
\begin{factlist}
\item{}
     {kirjoittaminen, tapahtumanjärjestäminen, käsityöt, ohjelmointi}
\end{factlist}


\end{document}
